% !TeX root = main.tex
%!TEX program = xelatex
\documentclass[11pt, aspectratio=169]{beamer}
\usefonttheme{professionalfonts}

\usepackage{amsmath}
\usepackage{fontspec}
\usepackage{graphicx}
\usepackage{import}
\usepackage{kotex}
\PassOptionsToPackage{table}{xcolor}
\usepackage{calc}
\usepackage{listings}
\usepackage{indentfirst}
\usepackage{tabularx}
\usepackage{ulem}
\usepackage{multicol}
\usepackage{epigraph}
\usepackage[many]{tcolorbox}

\definecolor{boj}{RGB}{0,118,191}
\definecolor{main-color}{RGB}{0,153,255}
\definecolor{acgreen}{RGB}{0,159,107}
\definecolor{wared}{RGB}{231,76,60}

\definecolor{acbronze}{RGB}{173,86,0}
\definecolor{acsilver}{RGB}{67,95,122}
\definecolor{acgold}{RGB}{236,154,0}
\definecolor{acplatinum}{RGB}{39,226,164}
\definecolor{acdiamond}{RGB}{0,180,252}
\definecolor{acruby}{RGB}{255,0,98}

\setbeamercolor{title}{fg=black}
\setbeamercolor{frametitle}{fg=main-color}
\setbeamercolor{structure}{fg=main-color}

\linespread{1.2}
\everymath{\displaystyle}

\graphicspath{ {./images/} }
\lstset{basicstyle=\footnotesize\ttfamily,breaklines=true}

\newcommand{\translation}[1]{\textsuperscript{#1}}

\setlength\fboxsep{0pt}

\newcommand{\complexity}[1]{$\mathcal{O}\left({#1}\right)$}
\newcommand{\difficulty}[1]{\includegraphics[width=1em,natwidth=1000,natheight=1000]{#1.svg.png}}
\newcommand{\norm}[1]{\left\lVert#1\right\rVert}

\usetheme{SSPC2025}
\usetikzlibrary{arrows.meta,matrix,decorations.pathreplacing}

\title{SSPC 2025 연습 세션 풀이}
\subtitle{Official Solutions}
\author{skuru}
\date{2025년 8월 15일}

\begin{document}
\setcounter{framenumber}{-1}
\frame{\titlepage}

\section{Practice}
\begin{frame}
    {\huge \addfontfeatures{LetterSpace=-5} \color{main-color} \textbf{연습 세션}}
    \vspace{3mm}
    \begin{center}
        \begin{tabular}{cl|l|l}
            \hline
            문제         &      & 의도한 난이도                         & 출제자            \\
            \hline
            \hline
            \textbf{A} & 덱일까 큐일까 & \textbf{\color{acbronze}Easy}   & \texttt{skuru} \\
            \textbf{B} & 사칙연산 & \textbf{\color{acbronze}Easy}   & \texttt{skuru} \\
            \textbf{C} & 약분하기 & \textbf{\color{acbronze}Easy}   & \texttt{skuru} \\
            \textbf{D} & 출제하기 & \textbf{\color{acgold}Medium} & \texttt{skuru} \\
            \textbf{E} & 마왕 토벌 & \textbf{\color{acplatinum}Hard} & \texttt{skuru} \\
            \hline
        \end{tabular}
    \end{center}
\end{frame}

\import{solutions/}{A-is-it-deque-or-queue.tex}
\import{solutions/}{B-elementary-arithmetic.tex}
\import{solutions/}{C-reducing-fractions.tex}
\import{solutions/}{D-setting.tex}
\import{solutions/}{E-subjugating-the-demon-lord.tex}
\end{document}
