% !TeX root = main.tex
%!TEX program = xelatex
\input{header}
\usetikzlibrary{arrows.meta,matrix,decorations.pathreplacing}

\title{SSPC 2025 연습 세션 풀이}
\subtitle{Official Solutions}
\author{skuru}
\date{2048년 2월 30일}

\begin{document}
\setcounter{framenumber}{-1}
\frame{\titlepage}

\section{Practice}
\begin{frame}
    {\huge \addfontfeatures{LetterSpace=-5} \color{main-color} \textbf{연습 세션}}
    \vspace{3mm}
    \begin{center}
        \begin{tabular}{cl|l|l}
            \hline
            문제         &      & 의도한 난이도                         & 출제자            \\
            \hline
            \hline
            \textbf{A} & 덱일까 큐일까 & \textbf{\color{acbronze}Easy}   & \texttt{skuru} \\
            \textbf{B} & 사칙연산 & \textbf{\color{acbronze}Easy}   & \texttt{skuru} \\
            \textbf{C} & 약분하기 & \textbf{\color{acbronze}Easy}   & \texttt{skuru} \\
            \textbf{D} & 출제하기 & \textbf{\color{acgold}Medium} & \texttt{skuru} \\
            \textbf{E} & 마왕 토벌 & \textbf{\color{acplatinum}Hard} & \texttt{skuru} \\
            \hline
        \end{tabular}
    \end{center}
\end{frame}

\import{solutions/}{A-is-it-deque-or-queue.tex}
\import{solutions/}{B-elementary-arithmetic.tex}
\import{solutions/}{C-reducing-fractions.tex}
\import{solutions/}{D-setting.tex}
\import{solutions/}{E-subjugating-the-demon-lord.tex}
\end{document}
