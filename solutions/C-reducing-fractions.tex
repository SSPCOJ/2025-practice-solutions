\section{C. 약분하기}

\begin{frame} % No title at first slide
    \sectiontitle{C}{약분하기}
    \sectionmeta{
        \texttt{math, euclidean}\\
        출제진 의도 -- \textbf{\color{acbronze}Easy}
    }
    \begin{itemize}
        \item 제출 -번, 정답 -명 (정답률 -\%)
        \item 처음 푼 사람: 없음, -분
        \item 출제자: \texttt{skuru}
    \end{itemize}
\end{frame}

\begin{frame}{\textbf{C}. 약분하기}
    \begin{itemize}
        \item 약분이란 분모와 분자를 그 둘의 공약수로 나누는 것과 같습니다.
        \item 따라서 기약분수가 되려면 분모와 분자의 최대공약수로 나누면 됩니다.
        \item 유클리드 호제법을 사용하면 \complexity{\log\max(p, q)}에 해결할 수 있습니다. \\[2em]
        \item 원래 해당 문제는 SSPC에 출제될 예정이었지만 파이썬의 gcd함수가 유클리드 호제법으로 구현되어 있는 관계로 연습 세션에 출제되었습니다.
    \end{itemize}
\end{frame}
